\documentclass[12pt]{article}
\usepackage{polski}
\usepackage[utf8]{inputenc}
\usepackage{fullpage}
\usepackage{tabto}
\usepackage{graphicx} 
\usepackage{float}
\usepackage{caption}
\usepackage{indentfirst}




\linespread{1.3}
\begin{document}
%---------------------------------------------------------
%					Strona Tytułowa
%---------------------------------------------------------
\begin{titlepage}
%-----------------------Tytuł-----------------------------
\newcommand{\LINE}{\rule{\linewidth}{0.7mm}}
\center
\LINE \\[0.5cm]
\Huge\textsc{Cyfrowa historia pojazdu}\\ [5mm]
\normalsize\textsc{Zastosowanie Informatyki w Gospodarce}  \\[0.5cm]
\LINE \\[3cm]
%----------------------Nazwiska---------------------------
\begin{minipage}{0.5\textwidth}
\begin{flushleft} \large
\emph{Autorzy:}
		\\Grzegorz Milaszkiewicz %226110
		\\Mateusz Ożóg %226125
		\\Igor Kurek
		\\Piotr Wodzyński
		\\Maciej Kucia 
\end{flushleft}
\end{minipage}
~
\begin{minipage}{0.45\textwidth}
\begin{flushright} \large
\emph{Prowadzący:} \\
dr inż. Marek Woda
\end{flushright}
\end{minipage}\\[2cm]
%----------------------Stopka-----------------------------
\vfill
\center Wrocław 2019
\end{titlepage}

%---------------------------------------------------------
%					Spis treści
%---------------------------------------------------------
\renewcommand{\contentsname}{Spis treści}
\tableofcontents
\newpage

%---------------------------------------------------------
%					Część pierwsza
%---------------------------------------------------------
\section{Temat projektu}
Uzasadnienie biznesowe (skąd pomysł / dlaczego / czy jest innowacyjny i w jakim zakresie)\\

Tematem projektu jest system umożliwiający dokumentowanie historii pojazdów. Obecnie rynek samochodów używanych jest kojarzony z tak zwanym "kręceniem liczników", czyli zmniejszaniem faktycznego przebiegu pojazdu w celu podwyższenia jego wartości. Usługa korekty licznika jest stosunkowo tania i trudno wykrywalna. Najbardziej na tego typu zabiegach cierpią uczciwi posiadacze samochodów, którzy mają trudność w sprzedarzy pojazdów z oryginalnym przebiegiem. System do dokumentowania historii pojazdów umożliwiłby uczciwym sprzedawcom na podniesienie wartości swoich pojazdów. Pomysł dokumentowania historii pojazdu nie jest pomysłem innowacyjnym. Na rynku jest wiele aplikacji umożliwiających zapisywanie historii pojazdów jednakże żadna z nich nie weryfikuje wprowadzanych danych. System\textit{ Cyfrowej historii pojazdu} będzie posiadał podobne funkcjonalności co inne aplikacje dostępne na rynku, lecz dodaktowko wymaga wprowadzenia przebiegu pojazdu przy każdym wpisie do bazy danych. Kolejną funkcjonalnością potwierdzającą dodanie wpisu będzie możliwość dodania zdjęcia (na przykład zdjęcia licznika pojazdu).




%---------------------------------------------------------
%					Część druga
%---------------------------------------------------------
\newpage
\section{Zakres projektu}
\subsection{Cele}
Celem projektu jest stworzenie systemu umożliwiającego dokumentowanie histroii pojazdów. Głównym zadaniem jest napisanie aplikacji mobilnej i webowej, która będzie przyjazna dla użytkownika. Aplikacja będzie przeznaczona dla właścicieli pojazdów oraz dla serwisów samochodowych. 
\subsection{Ryzyka}
Projektując taki system należy liczyć się z nieuniknionym ryzykiem. Głównym niebezpieczeństwem projektowanego systemu jest niskie lub całkowity brak zainteresowania ze strony użytkowników. Jest to dość realne zagrożenie z uwagi na mnogość tego typu aplikacji, które pomimo braku weryfikacji wprowadzanych informacji są o wiele bardziej wypromowane. \\

Kolejnym ryzykiem jest opóźnienie w wykonywaniu pracy. Jest to istotne niebezpieczeństwo, które może wyniknąć ze zbyt dużej ilości funkcjonalności oraz z niedostatecznej wiedzy programistycznej w wybranych technologiach. 
\subsection{Funkcjonalności podstawowe i rozszerzone}


\begin{table}[H]
\begin{center}
	\begin{tabular}{|p{0.18\linewidth}|p{0.72\linewidth}|}%{|l|l|}
	\hline
	Id wymagania 	& 1 				\\ \hline
	Nazwa			& Rejestracja konta \\ \hline
	Opis & Aplikacja powinna zarejestrować użytkownika po wypełnieniu
formularza rejestracyjnego\\ \hline
	Przesłanka & Możliwość dodawania nowych użytkowników  \\ \hline
	Ograniczenia i~warunki & Rejestracja nowych użytkowników może odbyć się tylko po potwierdzeniu adresu e-mail  \\ \hline
	Dane wejściowe & Typ konta: 1. Właściciel pojazdu, 2. Serwis \\
	&  1. Imię, nazwisko, rok urodzenia, email, hasło \\
	& 2. Nazwa serwisu, adres (ulica, kod pocztowy, miasto), adres email, hasło  \\ \hline
	Wynik & Dodanie nowego użytkownika do systemu \\ \hline
	Źródło danych wejściowych & Użytkownik (Właściciel pojazdu, Serwis samochodowy, Osoba zewnętrzna) \\ \hline
	Przeznaczenie & Każdy nowy użytkownik \\ \hline
	\end{tabular}
\captionof{table}{Karta wymagań dla funkcjonalności \textit{Rejestracja konta}} 
\end{center}
\end{table}



%---------------------------------------------------------
%					Część Trzecia
%---------------------------------------------------------
\newpage
\section{Aktualny stan rynku }
(jakie są podobne dostępne rozwiązania, co zostało w tej dziedzinie zrobione)
	




%---------------------------------------------------------
%					Część czwarta
%---------------------------------------------------------
\newpage
\section{Narzędzia i technologie zastosowane w projekcie}
(diagram elementów systemu)


%---------------------------------------------------------
%					Część piąta
%---------------------------------------------------------
\newpage
\section{Kamienie milowe i plan projektu}
\begin{itemize}
\item wykress Gantta
\item kalkulacja kosztów i nakładów pracy (rzeczywista)
\end{itemize}


%---------------------------------------------------------
%					Część szósta
%---------------------------------------------------------
\newpage
\section{Opis implementacji i wdrożenia projektu}
(kroki potrzebne by uruchomić / zainstalować projekt + wymagania sprzętowe i programowe)


%---------------------------------------------------------
%					Część siódma
%---------------------------------------------------------
\newpage
\section{Wnioski}




\end{document}


