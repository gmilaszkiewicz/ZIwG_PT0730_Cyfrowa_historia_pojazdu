\documentclass[12pt]{article}
\usepackage{polski}
\usepackage[utf8]{inputenc}
\usepackage{fullpage}
\usepackage{tabto}
\usepackage{graphicx} 
\usepackage{float}
\usepackage{caption}


\linespread{1.3}
\begin{document}
%---------------------------------------------------------
%					Strona Tytułowa
%---------------------------------------------------------
\begin{titlepage}
%-----------------------Tytuł-----------------------------
\newcommand{\LINE}{\rule{\linewidth}{0.7mm}}
\center
\LINE \\[0.5cm]
\Huge\textsc{Cyfrowa historia pojazdu}\\ [5mm]
\normalsize\textsc{Zastosowanie Informatyki w Gospodarce}  \\[0.5cm]
\LINE \\[3cm]
%----------------------Nazwiska---------------------------
\begin{minipage}{0.5\textwidth}
\begin{flushleft} \large
\emph{Autorzy:}
		\\Grzegorz Milaszkiewicz %226110
		\\Mateusz Ożóg %226125
		\\Igor Kurek
		\\Piotr Wodzyński
		\\Maciej Kucia 
\end{flushleft}
\end{minipage}
~
\begin{minipage}{0.45\textwidth}
\begin{flushright} \large
\emph{Prowadzący:} \\
dr inż. Marek Woda
\end{flushright}
\end{minipage}\\[2cm]
%----------------------Stopka-----------------------------
\vfill
\center Wrocław 2019
\end{titlepage}

%---------------------------------------------------------
%					Spis treści
%---------------------------------------------------------
\renewcommand{\contentsname}{Spis treści}
\tableofcontents
\newpage

%---------------------------------------------------------
%					Część pierwsza
%---------------------------------------------------------
\section{Temat projektu}
Uzasadnienie biznesowe (skąd pomysł / dlaczego / czy jest innowacyjny i w jakim zakresie)



%---------------------------------------------------------
%					Część druga
%---------------------------------------------------------
\newpage
\section{Zakres projektu}
\subsection{Cele}
\subsection{Ryzyka}
\subsection{Funkcjonalności podstawowe i rozszerzone}


%---------------------------------------------------------
%					Część Trzecia
%---------------------------------------------------------
\newpage
\section{Aktualny stan rynku }
(jakie są podobne dostępne rozwiązania, co zostało w tej dziedzinie zrobione)
	




%---------------------------------------------------------
%					Część czwarta
%---------------------------------------------------------
\newpage
\section{Narzędzia i technologie zastosowane w projekcie}
(diagram elementów systemu)


%---------------------------------------------------------
%					Część piąta
%---------------------------------------------------------
\newpage
\section{Kamienie milowe i plan projektu}
\begin{itemize}
\item wykress Gantta
\item kalkulacja kosztów i nakładów pracy (rzeczywista)
\end{itemize}


%---------------------------------------------------------
%					Część szósta
%---------------------------------------------------------
\newpage
\section{Opis implementacji i wdrożenia projektu}
(kroki potrzebne by uruchomić / zainstalować projekt + wymagania sprzętowe i programowe)


%---------------------------------------------------------
%					Część siódma
%---------------------------------------------------------
\newpage
\section{Wnioski}




\end{document}


